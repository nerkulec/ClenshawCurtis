\documentclass[11pt,wide]{mwart}
\usepackage[OT4,plmath]{polski}
\usepackage{graphicx}
\usepackage{empheq}
\usepackage{algorithm}
\usepackage{algorithmic}
\usepackage{biblatex}
\usepackage{amssymb}
\usepackage{hyperref}

\DeclarePairedDelimiter\ceil{\lceil}{\rceil}
\DeclarePairedDelimiter\floor{\lfloor}{\rfloor}

\begin{document}
\date{Wrocław, \today}
\title{\LARGE\textbf{Całka Clenshawa-Curtisa}\\Clenshaw-Curtis quadrature}
\author{Bartosz Brzoza\thanks{\textit{E-mail}: \texttt{309426@ii.uni.wroc.pl}}}
\maketitle

\section{Abstract}
We consider the problem of numerical integration.
A particular method called the Clenshaw-Curtis quadrature is presented,
explained and then evaluated. An efficient implementation in Julia
is also provided \href{https://github.com/nerkulec/ClenshawCurtis/}{(github)} as well as benchmarking code which shows off its effectiveness and limitations.
A brief discussion of numerical results follows.

\section{Specyfikacja problemu}
Mając funkcję $f(x)$ określoną na przedziale $[-1,1]$\\
chcemy tak dobrać węzły $x_1, ..., x_n$ oraz wagi $w_1, ..., w_n$, aby suma
$$\sum_{i=1}^n w_i f(x_i)$$
najlepiej przybliżała całkę
\begin{equation}
    \int_{-1}^1 f(x) dx \label{1}
\end{equation}

\section{Węzły równoodległe}
Przybliżanie całki \eqref{1} stosując metodę trapezów (lub metodę trapezów wyższych rzędów)
w węzłach równoodległych nie przynosi dobrych efektów dla wielomianów wysokiego stopnia.

\section{Algorytm Clenshawa-Curtisa}
Podstawiając $x = cos \theta $ do rozważanej całki:
$$\int_{-1}^1 f(x) dx = \int_0^\pi f(cos \theta) sin \theta d \theta$$
przekształcamy problem całkowania $f(x)$ na całkowanie $f(cos x) sin x$. 
Możemy to zrobić rozwijając funkcję $f(cos \theta)$ w szereg Fouriera:
$$f(cos \theta) = \sum_{k=0}^{\infty}{\vphantom{\sum}}' a_k cos(k \theta) \text{,}$$
Nie pojawiają się współczynniki z $sin(k\theta)$, gdyż funkcja $f(cos\theta)$ jest parzysta.
W powyższym wzorze współczynniki szeregu Fouriera wynoszą:
\begin{equation}
    a_k = \frac{2}{\pi} \int_0^\pi f(cos \theta) cos \theta d \theta \label{2}
\end{equation}
Te całki nadal musimy obliczyć numerycznie. Zauważmy, że funkcja $f(cos \theta)$ jest parzysta i okresowa.
Można zatem przybliżyć całkę \eqref{2} dyskretną transformacją kosinusową w $N+1$ równoodległych punktach
$\theta_n = \frac{n\pi}{N}$ $(n=0, ..., N)$:
$$a_k \approx \frac{2}{N}\left(\sum_{n=0}^{N}{\vphantom{\sum}}{''} f(\cos\frac{n\pi}N) \cos\frac{n k \pi}N \right)$$
którą można obliczyć w czasie $O(N log N)$ korzystając z algorytmu DCT typu I.\\
Zakłada on, że wartości w punktach ewaluacji po ekstrapolacji byłyby parzyste \\względem $n=0$, oraz parzyste względem $n=N$.\\
Ostatecznie całkę \eqref{1} można przedstawić:
$$\int_{-1}^1 f(x) dx = \int_0^\pi f(cos \theta) sin \theta d \theta \approx \sum_{k=0}^{N/2}{\vphantom{\sum}}{''} \frac{2 a_{2k}}{1-(2k)^2}$$

\section{Powiązanie z wielomianami Czebyszewa}
Zauważmy, że $T_k(cos \theta) = cos(k \theta)$, zatem szereg z poprzedniego działu to tak naprawdę
wyrażenie funkcji $f(x)$ w bazie wielomianów Czebyszewa:
$$f(x) = \sum_{k=0}^{\infty}{\vphantom{\sum}}' a_k T_k(x)$$
Zatem w istocie całkujemy przybliżenie funkcji $f(x)$ w bazie wielomianów Czebyszewa. Stosujemy węzły:
$$x_k = cos \frac{k \pi}N \text{ } (k=0, .., N)$$
które są ekstremami wielomianu Czebyszewa $T_N$.

\section{Przeprowadzone doświadczenia}
Wybrałem 4 klasy funkcji, na których testowane były powyższe metody:
\begin{enumerate}
    \item[1.] Funkcje postaci $f'(x)$, gdzie $f(x) = \sum_i \alpha_i exp(-(\frac{x-\mu_i}{\sigma_i})^2)$ (RBF)
    \item[2.] Funkcje postaci $f'(x)$, gdzie $f(x) = \sum_i \alpha_i cos(a_i x^3 + b_i x^2 + c_i x + d_i)$
    \item[3.] Funkcje przedziałami stałe o zbiorze wartości $\{0, 1\}$ (funkcje prostokątne)
    \item[4.] Funkcje będące sumą funkcji z pierwszego punktu i funkcji prostokątnej
\end{enumerate}
\subsection{Doświadczenia}
Dla każdej z klas wylosowałem po 1000 przykładów funkcji.\\
Za pomocą każdej z trzech metod obliczyłem całkę \eqref{1} oraz błąd względny kwadratury względem
wyniku analitycznego dla każdej z funkcji. Wykorzystywałem 1024 węzły dla każdej kwadratury.\\
Przedstawiam ilość cyfr dokładnych dla średniej oraz maksymalnej wartości błędu względnego:
\subsection{Wyniki}
\begin{center}
Ilość cyfr dokładnych dla średniego błędu\\
\begin{tabular}{ |c||c|c|c|c| } 
    \hline
    Ilość węzłów & RBF & Tryg-poly & Rect & RBF+Rect \\ 
    \hline
    \hline
    4 & -4.08 & -3.17 & -0.88 & -2.76 \\
    \hline
    16 & 0.75 & 14.56 & -0.77 & -1.26 \\
    \hline
    64 & 28.01 & 48.72 & -0.77 & -1.14 \\
    \hline
    256 & 48.99 & 49.0 & -0.77 & -1.12 \\
    \hline
    1024 & 49.6 & 49.11 & -0.77 & -1.12 \\
    \hline
\end{tabular}\\
Ilość cyfr dokładnych dla najgorszego błędu\\
\begin{tabular}{ |c||c|c|c|c| } 
\hline
Ilość węzłów & RBF & Tryg-poly & Rect & RBF+Rect \\ 
\hline
\hline

4 & -11.05 & -9.57 & -6.47 & -10.75 \\
\hline
16 & -7.21 & 8.41 & -6.2 & -9.18 \\
\hline
64 & 19.8 & 43.47 & -6.1 & -8.98 \\
\hline
256 & 41.28 & 43.71 & -6.11 & -8.9 \\
\hline
1024 & 42.44 & 43.85 & -6.11 & -8.9 \\
\hline
\end{tabular} 

\end{center}

\section{Wnioski}
\begin{enumerate}
\item Kwadratura Clenshawa-Curtisa dobrze przybliża całki funkcji, które są gładkie.
\item Ta kwadratura nie potrafi przybliżyć funkcji nieciągłych, nawet
gdy owe funkcje są przedziałami ciągłe.
\item Dużą zaletą tej metody całkowania jest możliwość wykonania obliczeń w czasie $O(n log n)$

\end{enumerate}
\end{document}